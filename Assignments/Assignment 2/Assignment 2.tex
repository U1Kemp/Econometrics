\documentclass[a4paper]{article}
\usepackage[a4paper, top=1in, left=1in]{geometry}
\usepackage{fancyhdr}
\usepackage{lastpage}
\usepackage{graphicx}
\usepackage{float}
\usepackage{amsmath}
\usepackage{amsfonts}
\usepackage{amssymb}
\usepackage{xcolor}
\usepackage{booktabs} % For table resizing

\pagestyle{fancy}
\fancyhf{}
\rhead{Econometrics Homework 2}
\lhead{Utpalraj Kemprai}

\cfoot{\thepage}

\begin{document}

\title{Econometrics Homework 2}
\author{Utpalraj Kemprai \\
MDS202352}
\date{\today}

\maketitle

\newpage

\section*{Question 1}

We have the following ordinal regression model:
\begin{align*}
    z_{i} &= x'_{i}\beta + \epsilon_{i} \quad \forall i = 1,\cdots,n \\
    \gamma_{j-1} &< z_{i} \leq \gamma_{j} \implies y_{i} = j, \quad \forall i, j = 1, \cdots, J
\end{align*}
where (in the first equation) $z_i$ is the latent variable for individual $i$, $x_i$ is a  vector of covariates,
$\beta$ is a $k\times1$ vector of unknown parameters, and $n$ denotes the number of observations. The second equation shows how $z_i$ is related to the observed discrete response $y_i$, where $-\infty = \gamma_0 < \gamma_1 <
\gamma_{J-1} < \gamma_J = \infty$ are the cut-points (or thresholds) and $y_i$ is assumed to have $J$ categories or outcomes.

\subsection*{(a)}
We assume that $\epsilon_i \sim N(0,1)$, for $i = 1,2,\cdots,n$. Therefore we have,
\begin{align*}
    Pr(y_i = j) &= Pr(\gamma_{j-1} < z_{i} \leq \gamma_j)\\
                &= Pr(\gamma_{j-1} < x'_{i}\beta + \epsilon_{i} \leq \gamma_j) \\
                &= Pr(\gamma_{j-1} - x'_{i}\beta < \epsilon_{i} \leq \gamma_j - x'_{i}\beta)\\
                &= \Phi(\gamma_j - x'_{i}\beta) - \Phi(\gamma_{j-1} - x'_{i}\beta) &[\text{where } \Phi(\cdot) \text{ is the cdf of }N(0,1) ]
\end{align*}

\end{document}