\documentclass[a4paper]{article}
\usepackage[a4paper, top=1in, left=1in]{geometry}
\usepackage{fancyhdr}
\usepackage{lastpage}
\usepackage{graphicx}
\usepackage{float}
\usepackage{amsmath}
\usepackage{amsfonts}
\usepackage{amssymb}
\usepackage{xcolor}
\usepackage{booktabs} % For table resizing
\usepackage{listings}

\lstset{
  language=R,             
  basicstyle=\ttfamily\small,   % Use a small monospaced font
  keywordstyle=\color{blue}, 
  commentstyle=\color{gray},  
  stringstyle=\color{red},  
  numbers=left,           
  numberstyle=\tiny,      
  stepnumber=1,           
  frame=single,           
  breaklines=true,        
  breakatwhitespace=true, 
  columns=flexible,       
  xleftmargin=5pt,        
  xrightmargin=5pt        
}


\pagestyle{fancy}
\fancyhf{}
\rhead{Econometrics Homework 2}
\lhead{Utpalraj Kemprai}

\cfoot{\thepage}

\begin{document}

\title{Econometrics Homework 2}
\author{Utpalraj Kemprai \\
MDS202352}
\date{\today}

\maketitle

\newpage

\section*{Question 1}

We have the following ordinal regression model:
\begin{align*}
    z_{i} &= x'_{i}\beta + \epsilon_{i} \quad \forall i = 1,\cdots,n \\
    \gamma_{j-1} &< z_{i} \leq \gamma_{j} \implies y_{i} = j, \quad \forall i, j = 1, \cdots, J
\end{align*}
where (in the first equation) $z_i$ is the latent variable for individual $i$, $x_i$ is a  vector of covariates,
$\beta$ is a $k\times1$ vector of unknown parameters, and $n$ denotes the number of observations. The second equation shows how $z_i$ is related to the observed discrete response $y_i$, where $-\infty = \gamma_0 < \gamma_1 <
\gamma_{J-1} < \gamma_J = \infty$ are the cut-points (or thresholds) and $y_i$ is assumed to have $J$ categories or outcomes.

\subsection*{(a)}

\subsubsection*{Probability of success}

We assume that $\epsilon_i \sim N(0,1)$, for $i = 1,2,\cdots, n$. Therefore we have,\\
The probability of success,
\begin{align*}
    Pr(y_i = j) &= Pr(\gamma_{j-1} < z_{i} \leq \gamma_j)\\
                &= Pr(\gamma_{j-1} < x'_{i}\beta + \epsilon_{i} \leq \gamma_j) \\
                &= Pr(\gamma_{j-1} - x'_{i}\beta < \epsilon_{i} \leq \gamma_j - x'_{i}\beta)\\
                &= \Phi(\gamma_j - x'_{i}\beta) - \Phi(\gamma_{j-1} - x'_{i}\beta) &[\text{where } \Phi(\cdot) \text{ is the cdf of }N(0,1) ]
\end{align*}

\subsubsection*{Likelihood function}

The likelihood function for the ordinal probit model is,
\begin{align*}
    L(\beta,\gamma;y) &= \prod_{i=1}^{n}\prod_{j=1}^{J} Pr(y_i = j | \beta,\gamma)^{I(y_i = j)} &[\text{where } I(\cdot) \text{ is the indicator function}]\\
                      &= \prod_{i=1}^{n}\prod_{j=1}^{J} (\Phi(\gamma_j - x'_{i}\beta) - \Phi(\gamma_{j-1} - x'_{i}\beta))^{I(y_i = j)}
\end{align*}

\subsection*{(b)}

\subsubsection*{Probability of success}

We assume that $\epsilon_i \sim \mathcal{L}(0,1)$, for $i = 1,2, \cdots, n$. Therefore the probability of success,
\begin{align*}
    Pr(y_i = j) &= Pr(\gamma_{j-1} < z_{i} \leq \gamma_j)\\
                &= Pr(\gamma_{j-1} < x'_{i}\beta + \epsilon_{i} \leq \gamma_j) \\
                &= Pr(\gamma_{j-1} - x'_{i}\beta < \epsilon_{i} \leq \gamma_j - x'_{i}\beta) \\
                &= \frac{1}{1+e^{-(\gamma_j - x'_{i}\beta)}} - \frac{1}{1+e^{-(\gamma_{j-1} - x'_{i}\beta)}} & [\text{as } \epsilon_i \sim \mathcal{L}(0,1)] \\
                &= \frac{e^{-(\gamma_{j-1} - x'_{i}\beta)} - e^{-(\gamma_{j} - x'_{i}\beta)}}{(1+e^{-(\gamma_j - x'_{i}\beta)})(1+e^{-(\gamma_{j-1} - x'_{i}\beta)})} \\
                &= \frac{e^{x'_i\beta}(e^{-\gamma_{j-1}} - e^{-\gamma_{j}})}{(1+e^{-(\gamma_j - x'_{i}\beta)})(1+e^{-(\gamma_{j-1} - x'_{i}\beta)})} 
\end{align*}

\subsubsection*{Likelihood function}

The likelihood function for the ordinal logit model is,
\begin{align*}
    L(\beta,\gamma;y) &= \prod_{i=1}^{n}\prod_{j=1}^{J} Pr(y_i = j | \beta, \gamma)^{I(y_i = j)} &[\text{where } I(\cdot) \text{ is the indicator function}]\\
                      &=\prod_{i=1}^{n}\prod_{j=1}^{J} [\frac{e^{x'_i\beta}(e^{-\gamma_{j-1}} - e^{-\gamma_{j}})}{(1+e^{-(\gamma_j - x'_{i}\beta)})(1+e^{-(\gamma_{j-1} - x'_{i}\beta)})}]^{I(y_i = j)} 
\end{align*}

\subsection*{(c)}

\subsubsection*{Probability remain unchanged on adding a constant c to cut-points and the mean}

Adding a constant $c$ to the cut-point $\gamma_j$ and the mean $x'_i\beta$, $\forall i = 1,2, \cdots, n$ and $j = 1,2,\cdots, J$, the latent variable $z_i$ becomes $x'_i\beta + c + \epsilon_i$ and the cut-point $\gamma_j$ becomes $\gamma_j + c$

The probability of $y_i$ taking the value $j$ is,
\begin{align*}
    Pr(y_i = j) &= Pr(\gamma_{j-1}+c < z_i \leq \gamma_j + c)\\
                &= Pr(\gamma_{j-1}+c < x'_i\beta + c + \epsilon_i \leq \gamma_j + c)\\
                &= Pr(\gamma_{j-1} - x'_i\beta  < \epsilon_i \leq \gamma_j - x'_i\beta ) \\
                &= \Phi(\gamma_j - x'_i\beta) - \Phi(\gamma_{j-1} - x'_i\beta)
\end{align*}
which is the same as the value obtained in part(a) of Question 1. So adding a constant c to the cut-point $\gamma_j$ and the mean $x'_i\beta$ does not change the outcome probability.

\subsubsection*{Identification Problem}
This identification problem can be solved by fixing the value of one of $\gamma_1, \gamma_2, \cdots, \gamma_{J-1}$. In particular setting $\gamma_1 = 0$ will solve this identification problem.

\subsection*{(d)}

\subsubsection*{Rescaling the parameters $(\gamma_j,\beta)$ and scale of distribution does not change outcome probability}

Rescaling the parameters $(\gamma_j,\beta)$ and the scale of the distribution of $\epsilon_i$ by some constant $d$, the latent variable $z_i$ becomes $x'_id\beta + \epsilon_i$ where $\epsilon_i \sim N(0,d^2)$ and the cut-point $\gamma_j$ becomes $d\gamma_j, \forall j = 1, 2, \cdots, J-1$.

The probability of $y_i$ taking the value $j$ is,
\begin{align*}
    Pr(y_i=j) &= Pr(d\gamma_{j-1} < z_i \leq d\gamma_j) \\
              &= Pr(d\gamma_{j-1} < x'_i d\beta + \epsilon_i \leq d\gamma_j)\\
              &= Pr(\gamma_{j-1} - x'_i\beta < \frac{\epsilon_i}{d} \leq \gamma_j - x'_i\beta)\\
              &= \Phi(\gamma_j - x'_i\beta) - \Phi(\gamma_{j-1} - x'_i\beta) & [\text{as } \epsilon_i \sim N(0,d^2), \frac{\epsilon_i}{d} \sim N(0,1)]
\end{align*}
which is the same as the value obtained in part(b) of Question 1. So escaling the parameters $(\gamma_j,\beta)$ and the scale of the distribution by some arbitrary constant $d$ lead to same outcome probabilities.

\subsubsection*{Identification problem}
This identification problem can be solved by fixing the scale of the distribution of $\epsilon_i$. In particular we can set the scale of the distribution of $\epsilon_i$ to 1, i.e. var($\epsilon_i$) = 1.

\pagebreak

\subsection*{(e)}
\begin{tiny}

\subsubsection*{(i) Descriptive Summary of the data}

\begin{table}[ht]
    \centering
    \begin{tabular}{@{}lcccc@{}}
    \toprule
    \text{VARIABLE} & \text{}       & \text{MEAN}       &\text{STD} \\ \midrule
    LOG AGE         &          &           3.72        &      0.44              \\
    LOG INCOME      &               &           10.63        &       0.98          \\ 
    HOUSEHOLD SIZE  &               &           2.74        &           1.42         \\
    \midrule
                    & \text{CATEGORY} & \text{COUNTS}  & \text{PERCENTAGE} \\ \midrule
    PAST USE        &               &      719             &        48.19            \\
    MALE        &               &       791            &             53.02       \\ 
                \\\cline{2-2}\\   
                &BACHELORS \& ABOVE        &     551     &     36.93               \\
    EDUCATION   &BELOW BACHELORS           &       434       &         29.09           \\
                &HIGH SCHOOL \& BELOW           &     507         &      33.98          \\
                \cline{2-2}\\   
    TOLERANT STATES   &          &       556            &       37.27             \\
    EVENTUALLY LEGAL   &          &     1154               &      77.35              \\
                \cline{2-2}\\   
                &WHITE       &     1149     &     77.01               \\
    RACE   &AFRICAN AMERICAN        &       202       &         13.54           \\
                &OTHER RACES           &     141         &      9.45          \\
                \cline{2-2}\\   
                &REPUBLICAN     &      333             &  22.32                  \\
    PARTY AFFILIATION   & DEMOCRAT           & 511           &34.25                    \\
                &INDEPENDENT \& OTHERS       &  648           &43.43                \\
                \cline{2-2}\\   
                &PROTESTANT       &     550              &36.86                    \\
                &ROMAN CATHOLIC        &290                   &19.44                    \\
    RELIGION    &CHRISTIAN           &182                   &12.20                    \\
                &CONSERVATIVE           &122                   &8.18                \\
                &LIBERAL        &  348                 &  23.32                  \\ 
                \cline{2-2}\\                     
                    &OPPOSE LEGALIZATION     &218        & 14.61                    \\
    PUBLIC OPINION  &LEGAL ONLY FOR MEDICINAL USE & 640      & 42.90                  \\
                    &LEGAL FOR PERSONAL USE      &634           &42.49              \\ \bottomrule
    \end{tabular}
    \caption{Descriptive Summary of the variables}
    \label{tab:Feb14Data}
    \end{table}

\end{tiny}

\pagebreak
\subsubsection*{(ii) Public opinion on extent of marijuana legalization}

We estimate Model 8 and replicate the results from the lecture.

% \begin{lstlisting}
% OrdProbit <- polr(factor(opinion) ~ log_age + log_income + hh1 + pastuse + 
%              bachelor_above + below_bachelor + tolerant_state + expected_legal +
%              black + other_race + democrat + other_party + male +
%              christian + roman_catholic + liberal + conservative,
%           data = dummy_copy, Hess = TRUE,
%           method = 'probit' # set to 'probit' for ordered probit
% )
%   \end{lstlisting}

% can t-value and the corresponding p-values
\begin{table}[ht]
    \centering
    \begin{tabular}{@{}lcccc@{}}
    \toprule
    \text{coef} & \text{Estimate}    & \text{Std. Error} &\text{t value} &\text{p value} \\ 
    \midrule
    intercept        & 0.35        &0.48    &0.72  &0.47\\
    log\_age          & -0.35      & 0.08   & -4.51 &  $< 0.05$\\
    log\_income        & 0.09      & 0.04    & 2.47 & $< 0.05$  \\
    hh1               &-0.02       &0.02      & -0.87  & 0.38     \\
    pastuse           & 0.69       &0.06      &10.67 & $< 0.05$\\
    bachelor\_above    & 0.24      & 0.08     & 3.01 & $< 0.05$\\
    below\_bachelor    & 0.05      & 0.08      &0.62  & 0.54\\
    tolerant\_state    & 0.07      & 0.07      &1.04   & 0.30\\
    expected\_legal   &  0.57      & 0.07      &7.76 & $< 0.05$\\
    black             & 0.03       &0.10        &0.26 & 0.79\\
    other\_race       & -0.28      & 0.11       &-2.56 & $< 0.05$\\
    democrat          & 0.44       & 0.09      &5.03 & $< 0.05$\\
    other\_party     &   0.36      & 0.08      &4.56 & $< 0.05$\\
    male              & 0.06       & 0.06      &1.00  & 0.32\\
    christian         & 0.16       & 0.10       &1.60 & 0.11 \\
    roman\_catholic   &  0.10      &  0.09      &1.19  & 0.23\\
    liberal           & 0.39       &0.09       &4.39 & $< 0.05$\\
    conservative      & 0.09       &0.12       &0.76 & 0.45\\    
    cut\_point         & 1.46       &0.05      &29.58 & $< 0.05$  \\\midrule 
    \\
    LR ($\chi^2$) Statistic       & 377        \\
    McFadden's $R^2$              & 0.13        \\
    Hit-Rate                      & 58.91        \\\bottomrule
    % $^{**} p < 0.05, ^{*}p < 0.10$
    \end{tabular}
    
    \caption{Model 8: Estimation Results}
\end{table}



\subsubsection*{(iii) Covariate effects of the variables}

\begin{table}[ht]
    \centering
    \begin{tabular}{@{}lcccc@{}}
    \toprule
    \text{Covariate} &$\Delta$ P(not legal)& $\Delta$ P(medicinal use) &$\Delta$ P(personal use)\\ \midrule
    Age, 10 years    &  0.015          &  0.012          &  0.028   \\
    Income, \$ 10,000    &   -0.005         &   -0.003         &  0.008   \\
    Past Use    &    -0.129        &-0.113&    0.243 \\
    Bachelors \& Above    &   -0.045         &   -0.35         &  0.080   \\
    Eventually Legal    &       -0.126     &  -0.60          & 0.186    \\
    Other Races    &   0.059         &    0.031        & -0.089    \\
    Democrat    &     -0.80      & -0.066          &   0.147  \\
    Other parties    &   -0.070         &   -0.051         &  0.121   \\
    Liberal    &   -0.68         &      -0.066      &   0.134  \\\bottomrule
    \end{tabular}
    
    \caption{Average covariate effects from Model 8.}
\end{table}

\newpage
\section*{Question 2}

\subsection*{(a)}

\subsection*{(b)}

\subsection*{(c)}
\end{document}