\documentclass[a4paper]{article}
\usepackage[a4paper, top=1in, left=1in]{geometry}
\usepackage{fancyhdr}
\usepackage{lastpage}
\usepackage{graphicx}
\usepackage{float}
\usepackage{amsmath}
\usepackage{amsfonts}
\usepackage{amssymb}
\usepackage{xcolor}
\usepackage{booktabs} % For table resizing
\usepackage{listings}
\usepackage{titling}
\renewcommand\maketitlehooka{\null\mbox{}\vfill}
\renewcommand\maketitlehookd{\vfill\null}

\lstset{
  language=R,
  frame=single,
  basicstyle=\ttfamily\footnotesize,
  columns=flexible,
  numbers=left,
  numberstyle=\tiny,
  stepnumber=1,
  numbersep=10pt,
  captionpos=b,
  backgroundcolor=\color{white},
%   keywordstyle=\color{blue}\bfseries,     % Keywords in blue and bold
  stringstyle=\color{red},                % Strings in red
  commentstyle=\color{blue},             % Comments in green
%   morekeywords={plm,summary,pFtest},      % Highlight these custom keywords
%   deletekeywords={data,for,effects},                 % Remove "data" from the list of keywords
%   showstringspaces=false           % Do not underline spaces in strings
}

\pagestyle{fancy}
\fancyhf{}
\rhead{Econometrics Homework 2}
\lhead{Utpalraj Kemprai}

\cfoot{\thepage}

\begin{document}

\title{\huge Econometrics Homework 2}
\author{\LARGE Utpalraj Kemprai \\[5pt]
\LARGE MDS202352}
\date{}

\begin{titlingpage}
    \maketitle
\end{titlingpage}

\newpage

\section*{Question 1}

We have the following ordinal regression model:
\begin{align*}
    z_{i} &= x'_{i}\beta + \epsilon_{i} \quad \forall i = 1,\cdots,n \\
    \gamma_{j-1} &< z_{i} \leq \gamma_{j} \implies y_{i} = j, \quad \forall i, j = 1, \cdots, J
\end{align*}
where (in the first equation) $z_i$ is the latent variable for individual $i$, $x_i$ is a  vector of covariates,
$\beta$ is a $k\times1$ vector of unknown parameters, and $n$ denotes the number of observations. The second equation shows how $z_i$ is related to the observed discrete response $y_i$, where $-\infty = \gamma_0 < \gamma_1 <
\gamma_{J-1} < \gamma_J = \infty$ are the cut-points (or thresholds) and $y_i$ is assumed to have $J$ categories or outcomes.

\subsection*{(a)}

\subsubsection*{Probability of success}

We assume that $\epsilon_i \sim N(0,1)$, for $i = 1,2,\cdots, n$. Therefore we have,\\
The probability of success,
\begin{align*}
    Pr(y_i = j) &= Pr(\gamma_{j-1} < z_{i} \leq \gamma_j)\\
                &= Pr(\gamma_{j-1} < x'_{i}\beta + \epsilon_{i} \leq \gamma_j) \\
                &= Pr(\gamma_{j-1} - x'_{i}\beta < \epsilon_{i} \leq \gamma_j - x'_{i}\beta)\\
                &= \Phi(\gamma_j - x'_{i}\beta) - \Phi(\gamma_{j-1} - x'_{i}\beta) &[\text{where } \Phi(\cdot) \text{ is the cdf of }N(0,1) ]
\end{align*}

\subsubsection*{Likelihood function}

The likelihood function for the ordinal probit model is,
\begin{align*}
    L(\beta,\gamma;y) &= \prod_{i=1}^{n}\prod_{j=1}^{J} Pr(y_i = j | \beta,\gamma)^{I(y_i = j)} &[\text{where } I(\cdot) \text{ is the indicator function}]\\
                      &= \prod_{i=1}^{n}\prod_{j=1}^{J} \Big(\Phi(\gamma_j - x'_{i}\beta) - \Phi(\gamma_{j-1} - x'_{i}\beta)\Big)^{I(y_i = j)}
\end{align*}

\subsection*{(b)}

\subsubsection*{Probability of success}

We assume that $\epsilon_i \sim \mathcal{L}(0,1)$, for $i = 1,2, \cdots, n$. Therefore the probability of success,
\begin{align*}
    Pr(y_i = j) &= Pr(\gamma_{j-1} < z_{i} \leq \gamma_j)\\
                &= Pr(\gamma_{j-1} < x'_{i}\beta + \epsilon_{i} \leq \gamma_j) \\
                &= Pr(\gamma_{j-1} - x'_{i}\beta < \epsilon_{i} \leq \gamma_j - x'_{i}\beta) \\
                &= \frac{1}{1+e^{-(\gamma_j - x'_{i}\beta)}} - \frac{1}{1+e^{-(\gamma_{j-1} - x'_{i}\beta)}} & [\text{as } \epsilon_i \sim \mathcal{L}(0,1)] \\
                &= \frac{e^{-(\gamma_{j-1} - x'_{i}\beta)} - e^{-(\gamma_{j} - x'_{i}\beta)}}{(1+e^{-(\gamma_j - x'_{i}\beta)})(1+e^{-(\gamma_{j-1} - x'_{i}\beta)})} \\
                &= \frac{e^{x'_i\beta}(e^{-\gamma_{j-1}} - e^{-\gamma_{j}})}{(1+e^{-(\gamma_j - x'_{i}\beta)})(1+e^{-(\gamma_{j-1} - x'_{i}\beta)})} 
\end{align*}

\subsubsection*{Likelihood function}

The likelihood function for the ordinal logit model is,
\begin{align*}
    L(\beta,\gamma;y) &= \prod_{i=1}^{n}\prod_{j=1}^{J} Pr(y_i = j | \beta, \gamma)^{I(y_i = j)} &[\text{where } I(\cdot) \text{ is the indicator function}]\\
                      &= \displaystyle \prod_{i=1}^{n}\prod_{j=1}^{J} \left(\frac{e^{x'_i\beta}(e^{-\gamma_{j-1}} - e^{-\gamma_{j}})}{(1+e^{-(\gamma_j - x'_{i}\beta)})(1+e^{-(\gamma_{j-1} - x'_{i}\beta)})}\right)^{I(y_i = j)}
\end{align*}

\subsection*{(c)}

\subsubsection*{Probability remain unchanged on adding a constant c to cut-points and the mean}

If we add a constant $c$ to the cut-points $\gamma_j$ and the means $x'_i\beta$, then $\forall j = 1,2,\cdots, J-1$ and $\forall i = 1,2,\cdots, n$,  the latent variable $z_i$ becomes $x'_i\beta + c + \epsilon_i$ and the cut-point $\gamma_j$ becomes $\gamma_j + c$.
\vspace{0.25cm}

\noindent The probability of $y_i$ taking the value $j$ is,
\begin{align*}
    Pr(y_i = j) &= Pr(\gamma_{j-1}+c < z_i \leq \gamma_j + c)\\
                &= Pr(\gamma_{j-1}+c < x'_i\beta + c + \epsilon_i \leq \gamma_j + c)\\
                &= Pr(\gamma_{j-1} - x'_i\beta  < \epsilon_i \leq \gamma_j - x'_i\beta ) \\
                &= \Phi(\gamma_j - x'_i\beta) - \Phi(\gamma_{j-1} - x'_i\beta)
\end{align*}
which is the same as the value obtained in part(a) of Question 1. So adding a constant c to the cut-point $\gamma_j$ and the mean $x'_i\beta$ does not change the outcome probability.

\subsubsection*{Identification Problem}
This identification problem can be solved by fixing the value of one of $\gamma_1, \gamma_2, \cdots, \gamma_{J-1}$. In particular setting $\gamma_1 = 0$ will solve this identification problem.

\subsection*{(d)}

\subsubsection*{Rescaling the parameters $(\gamma_j,\beta)$ and scale of distribution does not change outcome probability}

Rescaling the parameters $(\gamma_j,\beta)$ and the scale of the distribution of $\epsilon_i$ by some constant $d > 0$, the latent variable $z_i$ becomes $x'_id\beta + \epsilon_i$ where $\epsilon_i \sim N(0,d^2)$, $\forall i = 1,2,\cdots,n$ and the cut-point $\gamma_j$ becomes $d\gamma_j, \forall j = 1, 2, \cdots, J-1$.
\vspace{0.25cm}

\noindent The probability of $y_i$ taking the value $j$ is,
\begin{align*}
    Pr(y_i=j) &= Pr(d\gamma_{j-1} < z_i \leq d\gamma_j) \\
              &= Pr(d\gamma_{j-1} < x'_i d\beta + \epsilon_i \leq d\gamma_j)\\
              &= Pr(\gamma_{j-1} - x'_i\beta < \frac{\epsilon_i}{d} \leq \gamma_j - x'_i\beta)\\
              &= \Phi(\gamma_j - x'_i\beta) - \Phi(\gamma_{j-1} - x'_i\beta) & [\text{as } \epsilon_i \sim N(0,d^2), \frac{\epsilon_i}{d} \sim N(0,1)]
\end{align*}
which is the same as the value obtained in part(a) of Question 1. So rescaling the parameters $(\gamma_j,\beta)$ and the scale of the distribution by some arbitrary constant $d$ lead to same outcome probabilities.

\subsubsection*{Identification problem}
This identification problem can be solved by fixing the scale of the distribution of $\epsilon_i$. In particular we can set the scale of the distribution of $\epsilon_i$ to 1, i.e. var($\epsilon_i$) = 1.

\pagebreak

\subsection*{(e)}
\begin{tiny}

\subsubsection*{(i) Descriptive Summary of the data}

\begin{table}[ht]
    \centering
    \begin{tabular}{@{}lcccc@{}}
    \toprule
    \text{VARIABLE} & \text{}       & \text{MEAN}       &\text{STD} \\ \midrule
    LOG AGE         &          &           3.72        &      0.44              \\
    LOG INCOME      &               &           10.63        &       0.98          \\ 
    HOUSEHOLD SIZE  &               &           2.74        &           1.42         \\
    \midrule
                    & \text{CATEGORY} & \text{COUNTS}  & \text{PERCENTAGE} \\ \midrule
    PAST USE        &               &      719             &        48.19            \\
    MALE        &               &       791            &             53.02       \\ 
                \\\cline{2-2}\\   
                &BACHELORS \& ABOVE        &     551     &     36.93               \\
    EDUCATION   &BELOW BACHELORS           &       434       &         29.09           \\
                &HIGH SCHOOL \& BELOW           &     507         &      33.98          \\
                \cline{2-2}\\   
    TOLERANT STATES   &          &       556            &       37.27             \\
    EVENTUALLY LEGAL   &          &     1154               &      77.35              \\
                \cline{2-2}\\   
                &WHITE       &     1149     &     77.01               \\
    RACE   &AFRICAN AMERICAN        &       202       &         13.54           \\
                &OTHER RACES           &     141         &      9.45          \\
                \cline{2-2}\\   
                &REPUBLICAN     &      333             &  22.32                  \\
    PARTY AFFILIATION   & DEMOCRAT           & 511           &34.25                    \\
                &INDEPENDENT \& OTHERS       &  648           &43.43                \\
                \cline{2-2}\\   
                &PROTESTANT       &     550              &36.86                    \\
                &ROMAN CATHOLIC        &290                   &19.44                    \\
    RELIGION    &CHRISTIAN           &182                   &12.20                    \\
                &CONSERVATIVE           &122                   &8.18                \\
                &LIBERAL        &  348                 &  23.32                  \\ 
                \cline{2-2}\\                     
                    &OPPOSE LEGALIZATION     &218        & 14.61                    \\
    PUBLIC OPINION  &LEGAL ONLY FOR MEDICINAL USE & 640      & 42.90                  \\
                    &LEGAL FOR PERSONAL USE      &634           &42.49              \\ \bottomrule
    \end{tabular}
    \caption{Descriptive Summary of the variables}
    \label{tab:Feb14Data}
    \end{table}

\end{tiny}

\pagebreak
\subsubsection*{(ii) Public opinion on extent of marijuana legalization}

We estimate Model 8 and replicate the results from the lecture.

\begin{table}[ht]
    \centering
    \begin{tabular}{@{}lrrrr@{}}
    \toprule
    \text{coef} & \text{Estimate}    & \text{Std. Error} &\text{t value} &\text{p value} \\ 
    \midrule
    intercept        & 0.35        &0.48    &0.72  &0.47\\
    log\_age          & -0.35      & 0.08   & -4.51 &  $< 0.05$\\
    log\_income        & 0.09      & 0.04    & 2.47 & $< 0.05$  \\
    hh1(household size) &-0.02       &0.02      & -0.87  & 0.38     \\
    pastuse           & 0.69       &0.06      &10.67 & $< 0.05$\\
    bachelor\_above    & 0.24      & 0.08     & 3.01 & $< 0.05$\\
    below\_bachelor    & 0.05      & 0.08      &0.62  & 0.54\\
    tolerant\_state    & 0.07      & 0.07      &1.04   & 0.30\\
    expected\_legal   &  0.57      & 0.07      &7.76 & $< 0.05$\\
    black             & 0.03       &0.10        &0.26 & 0.79\\
    other\_race       & -0.28      & 0.11       &-2.56 & $< 0.05$\\
    democrat          & 0.44       & 0.09      &5.03 & $< 0.05$\\
    other\_party     &   0.36      & 0.08      &4.56 & $< 0.05$\\
    male              & 0.06       & 0.06      &1.00  & 0.32\\
    christian         & 0.16       & 0.10       &1.60 & 0.11 \\
    roman\_catholic   &  0.10      &  0.09      &1.19  & 0.23\\
    liberal           & 0.39       &0.09       &4.39 & $< 0.05$\\
    conservative      & 0.09       &0.12       &0.76 & 0.45\\    
    cut\_point($\gamma_1$)         & 1.46       &0.05      &29.58 & $< 0.05$  \\\midrule 
    \\
    LR ($\chi^2$) Statistic       & 377        \\
    McFadden's $R^2$              & 0.13        \\
    Hit-Rate                      & 58.91        \\\bottomrule
    % $^{**} p < 0.05, ^{*}p < 0.10$
    \end{tabular}
    
    \caption{Model 8: Estimation Results}
\end{table}



\subsubsection*{(iii) Covariate effects of the variables}

\begin{table}[ht]
    \centering
    \begin{tabular}{@{}lcccc@{}}
    \toprule
    \text{Covariate} &$\Delta$ P(not legal)& $\Delta$ P(medicinal use) &$\Delta$ P(personal use)\\ \midrule
    Age, 10 years    &  0.015          &  0.012          &  -0.028   \\
    Income, \$ 10,000    &   -0.005         &   -0.003         &  0.008   \\
    Past Use    &    -0.129        &-0.113&    0.243 \\
    Bachelors \& Above    &   -0.045         &   -0.035         &  0.080   \\
    Eventually Legal    &       -0.126     &  -0.060          & 0.186    \\
    Other Races    &   0.059         &    0.031        & -0.089    \\
    Democrat    &     -0.080      & -0.066          &   0.147  \\
    Other parties    &   -0.070         &   -0.051         &  0.121   \\
    Liberal    &   -0.068         &      -0.066      &   0.134  \\\bottomrule
    \end{tabular}
    
    \caption{Average covariate effects from Model 8.}
\end{table}

\newpage
\section*{Question 2}

\subsection*{Grunfeld Investment Study}

Investment is modeled as a function of market value, capital, and firm.
There are 200 observation on 10 firms from 1935-1954 (20 years). We have excluded the data on the company American Steel.

\subsubsection*{Variable Description}
\begin{itemize}
    \item invest: Gross investment in 1947 dollars.
    \item value: Market value as of Dec. 31 in 1947 dollars.
    \item capital: Stock of plant and equipment in 1947 dollars.
    \item firm: General Motors, US Steel, General Electric, Chrysler, Atlantic Refining, IBM, Union Oil, Westinghouse, Goodyear, Diamond Match.
\end{itemize}

\subsection*{(a)}
\textbf{Pooled Effects model}
\begin{table}[ht]
    \centering
    \begin{tabular}{@{}lrrrrr@{}}
        \toprule
                & Estimate  & Std. Error  & t-value & \(\text{Pr}(>|t|)\)\\
                \midrule
        intercept & -42.71 & 9.511 & -4.49 & \(<0.001\)\\
        capital   & 0.23   & 0.025 & 9.05  & \(<0.001\)\\
        value     & 0.12   & 0.006 & 19.80 & \(<0.001\)\\
        \bottomrule
    \end{tabular}
    \caption{Covariates of the Pooled Effects model}
\end{table}
\begin{itemize}
    \item R-Squared: 0.812
    \item Adj. R-Squared: 0.811
    \item F-Statistic: 426.58 with 2 and 197 df, p-value \(<2.22 \times 10^{-16}\) 
\end{itemize}

\textbf{Interpretation of Coefficients}
\begin{itemize}
    \item capital: It's coefficient is statistically significant(\(\text{p-value}<0.001\)) and has a positive value (0.23). So investment increases with increase in capital as per our Pooled Effects model.
    \item value: It's coefficient is also statistically significant(\(\text{p-value}<0.001\)) and is positive (0.12). So a increase in value also increases the investment, but relatively lesser compared to capital. 
\end{itemize}

The model is able to explain about \(81.2\%\) variance in the data and the high value of the F-Statistic also shows that the variables are statistically significant.

\newpage
\subsection*{(b)}
\textbf{Fixed Effects model}
\begin{table}[ht]
    \centering
    \begin{tabular}{@{}lrrrrr@{}}
        \toprule
                & Estimate  & Std. Error  & t-value & \(\text{Pr}(>|t|)\)\\
                \midrule
        capital   & 0.31   & 0.017 & 17.87  & \(<0.001\)\\
        value     & 0.11   & 0.012 & 9.29 & \(<0.001\)\\
        \bottomrule
    \end{tabular}
    \caption{Covariates of the Fixed Effects model}
\end{table}

\begin{table}[ht]
    \centering
    \begin{tabular}{lrrrr}
    \toprule
     & {Estimate} & {Std. Error} & {t-value} & \(\text{Pr}(>|t|)\) \\ \midrule
    Atlantic Refining   & -114.62 & 14.17 & -8.09  & $<0.001$ \\ 
    Chrysler            & -27.81  & 14.08 & -1.98  & 0.05  \\ 
    Diamond Match       & -6.57   & 11.83 & -0.56  & 0.58 \\ 
    General Electric    & -235.57 & 24.43 & -9.64  & $<0.001$ \\ 
    General Motors      & -70.30  & 49.71 & -1.41  & 0.16 \\ 
    Goodyear            & -87.22  & 12.89 & -6.77  & $<0.001$ \\ 
    IBM                 & -23.16  & 12.67 & -1.83  & 0.07  \\ 
    Union Oil           & -66.55  & 12.84 & -5.18  & $<0.001$  \\ 
    US Steel            & 101.91  & 24.94 & 4.09   & $<0.001$  \\ 
    Westinghouse        & -57.55  & 13.99 & -4.11  & $<0.001$  \\
    \bottomrule
    \end{tabular}
    \caption{Firm specific intercepts}
\end{table}
    
\begin{itemize}
    \item R-Squared: 0.767
    \item Adj. R-Squared: 0.753
    \item F-Statistic: 309.01 with 2 and 188 df, p-value \(<2.22 \times 10^{-16}\) 
\end{itemize}

\noindent\textbf{Test for Individual Effects}

\begin{lstlisting}[caption={R Code for Test for Individual Effects}]
#Test for Individual Effects
#-----------------------------------------------------------------------------
pFtest(fe_model_plm, pooled_effect_plm)

# OUTPUT:
# F test for individual effects
# 
# data:  invest ~ capital + value
# F = 49.177, df1 = 9, df2 = 188, p-value < 2.2e-16
# alternative hypothesis: significant effects
\end{lstlisting}

As the p-value is very small(\(<2.2\times10^{-16}\)), the null hypothesis is rejected in favor of the alternative that there are significant fixed effects. Hence, a Fixed Effect model will be a better choice than a Pooled Effect model, in this case.


\subsubsection*{\textbf{Firm--Specific Intercepts}}
The firm-specific intercepts reveal interesting patterns in baseline investment behavior across companies. US Steel stands out with a significantly positive intercept (101.91), suggesting this firm maintains higher baseline investment levels compared to others, even after accounting for capital and value effects. In contrast, General Electric shows the most negative intercept (-235.57), indicating substantially lower baseline investment. The statistical significance of most intercepts (6 out of 10 firms showing p-values \(<\) 0.001) reinforces the importance of controlling for firm-specific effects when modeling investment behavior.

\subsubsection*{\textbf{Interpretation of Coefficients}}
\begin{itemize}
    \item capital:  It's coefficient is statistically significant(\(\text{p-value}<0.001\)) and has a positive value (0.31). So investment increases with increase in capital as per our Fixed Effects model.
    
    \item value: It's coefficient is also statistically significant(\(\text{p-value}<0.001\)) and is positive (0.11). So a increase in value also increases the investment, but relatively lesser compared to capital. 

\end{itemize}

% The model is able to explain about \(76.7\%\) of the variance in the data and the high value of the F-Statistic also shows that the variables are statistically significant. However the Fixed Effects model explains only within-firm variation only, which may not be directly comparable to the R-squared of the Pooled Effects model.

% \newpage
\subsection*{(c)}
\textbf{Random Effects model}
\begin{table}[ht]
    \centering
    \begin{tabular}{lrrr}
        \toprule
            & var & std.dev & share \\
            \midrule
    idiosyncratic & 2784.46 & 52.77 & 0.282 \\
    individual    & 7089.80 & 84.20 & 0.718 \\
    \bottomrule
    \(\theta\) = 0.861\\
    % \bottomrule        
    \end{tabular}    
    \caption{Variance Components of the Random Effects Model}
\end{table}
\begin{table}[ht]
    \centering
    \begin{tabular}{@{}lrrrrr@{}}
        \toprule
                & Estimate  & Std. Error  & z-value & \(\text{Pr}(>|z|)\)\\
                \midrule
        intercept & -57.83 & 28.90 & -2.00 & 0.045 \\
        capital   & 0.31   & 0.017 & 17.93  & \(<0.001\)\\
        value     & 0.11   & 0.010 & 10.46 & \(<0.001\)\\
        \bottomrule
    \end{tabular}
    \caption{Covariates of the Random Effects model}
\end{table}
\begin{itemize}
    \item R-Squared: 0.770
    \item Adj. R-Squared: 0.767
    \item \(\chi^2\) Statistic: 657.67 on 2 df, p-value \(<2.22 \times 10^{-16}\) 
\end{itemize}

\noindent\textbf{Hausman Test}

\begin{lstlisting}[caption={R Code for Hausman Test}]
# Hausman Test
#-----------------------------------------------------------------------------
phtest(fe_model_plm,re_model_plm)

# OUTPUT:
# Hausman Test
# 
# data:  invest ~ capital + value
# chisq = 2.3304, df = 2, p-value = 0.3119
# alternative hypothesis: one model is inconsistent
\end{lstlisting}

The null hypothesis cannot be rejected here (\(\text{p-value} = 0.3119\)). Hence it makes sense to use a Random Effects model instead of a Fixed Effects model.

\subsubsection*{\textbf{Idiosyncratic and Individual Variances}}
\begin{itemize}
    \item Idiosyncratic variance: Idiosyncratic variance refers to the within-firm variation. This captures how a firm's values change over time, representing the deviation from that firm's own average. This is essentially the variance of the error term after accounting for the firm-specific effects.
    \item Individual variance: Individual variance refers to the between-firm variation. This captures how different firms vary from each other in their average level of the dependent variable (investment in your case). It represents the variation in the random intercepts across different firms.
\end{itemize}

In our Random Effects model results, the individual variance (7089.80) being much larger than the idiosyncratic variance (2784.46) tells us that differences between firms explain more of the variation in investment than changes within firms over time. The share values confirm this: 71.8\% of the total variance comes from between-firm differences, while only 28.2\% comes from within-firm variation over time.

The high theta value (\(\theta\) = 0.861) further indicates the relative importance of the variation between firms in the total variation. 

\subsubsection*{\textbf{Interpretation of Coefficients}}
\begin{itemize}
    \item capital: It's coefficient is statistically significant (\(\text{p-value}<0.001\)) and is positive (0.31). So increase in capital increases the investment as per our Random Effects model.
    \item value: It's coefficient is also statistically significant (\(\text{p-value}<0.001\) and is positive (0.11). So increase in value also increases the investment but has a smaller effect than capital.
\end{itemize}

% The R-Squared for the Random Effects model is similar to the Fixed Effects model (0.770). It accounts for both the within-firm and between-firm variation, so is not directly comparable with the R-Squared of the Fixed Effects model and Pooled Effects model. Moreover \(86.1\%\) (\(\theta = 0.861\)) of the variation in the dependent variable is due to individual-specific effects rather than the idiosyncratic error term.

\end{document}